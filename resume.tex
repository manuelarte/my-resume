%%%%%%%%%%%%%%%%%%%%%%%%%%%% Document Setup %%%%%%%%%%%%%%%%%%%%%%%%%%%%

\documentclass[10pt]{article}

% This is a helpful package that puts math inside length specifications
\usepackage{calc}

\usepackage[utf8]{inputenc}
\usepackage[spanish]{babel}
\usepackage{graphicx}
\usepackage[usenames,dvipsnames]{xcolor}
\definecolor{letraE}{RGB}{140,89,93}
\definecolor{letraD}{RGB}{119,121,113}
\definecolor{letraX}{RGB}{62,125,148}
\definecolor{udacity}{RGB}{210,105,51}
%\definecolor{gray}{RGB}{80,80,80}
\definecolor{light-gray}{gray}{0.60}
\newcommand{\linkedin}{ \textit{\colorbox{gray}{\textcolor{white}{\textbf{Linked}}}\negthinspace\colorbox{cyan}{\textcolor{white}{\textbf{in}}}} }
\newcommand{\twitter}{ \textit{\colorbox{cyan}{\textcolor{white}{\textbf{Twitter}}}} }
\newcommand{\edX}{\textbf{\textcolor{letraE}{e}\textcolor{letraD}{d}\negthinspace\textcolor{letraX}{X}}}

% Simpler bibsection for CV sections
\makeatletter
\newlength{\bibhang}
\setlength{\bibhang}{1em}
\newlength{\bibsep}
 {\@listi \global\bibsep\itemsep \global\advance\bibsep by\parsep}
\newenvironment{bibsection}%
        {\vspace{-\baselineskip}\begin{list}{}{%
       \setlength{\leftmargin}{\bibhang}%
       \setlength{\itemindent}{-\leftmargin}%
       \setlength{\itemsep}{\bibsep}%
       \setlength{\parsep}{\z@}%
        \setlength{\partopsep}{0pt}%
        \setlength{\topsep}{0pt}}}
        {\end{list}\vspace{-.6\baselineskip}}
\makeatother

% Layout: Puts the section titles on left side of page
\reversemarginpar

%
%         PAPER SIZE, PAGE NUMBER, AND DOCUMENT LAYOUT NOTES:
%
% The next \usepackage line changes the layout for CV style section
% headings as marginal notes. It also sets up the paper size as either
% letter or A4. By default, letter was used. If A4 paper is desired,
% comment out the letterpaper lines and uncomment the a4paper lines.
%
% As you can see, the margin widths and section title widths can be
% easily adjusted.
%
% ALSO: Notice that the includefoot option can be commented OUT in order
% to put the PAGE NUMBER *IN* the bottom margin. This will make the
% effective text area larger.
%
% IF YOU WISH TO REMOVE THE ``of LASTPAGE'' next to each page number,
% see the note about the +LP and -LP lines below. Comment out the +LP
% and uncomment the -LP.
%
% IF YOU WISH TO REMOVE PAGE NUMBERS, be sure that the includefoot line
% is uncommented and ALSO uncomment the \pagestyle{empty} a few lines
% below.
%

%% Use these lines for letter-sized paper
\usepackage[paper=letterpaper,
            %includefoot, % Uncomment to put page number above margin
            marginparwidth=1.2in,     % Length of section titles
            marginparsep=.05in,       % Space between titles and text
            margin=1in,               % 1 inch margins
            includemp]{geometry}

%% Use these lines for A4-sized paper
%\usepackage[paper=a4paper,
%            %includefoot, % Uncomment to put page number above margin
%            marginparwidth=30.5mm,    % Length of section titles
%            marginparsep=1.5mm,       % Space between titles and text
%            margin=25mm,              % 25mm margins
%            includemp]{geometry}

%% More layout: Get rid of indenting throughout entire document
\setlength{\parindent}{0in}

%% This gives us fun enumeration environments. compactitem will be nice.
\usepackage{paralist}

%% Reference the last page in the page number
%
% NOTE: comment the +LP line and uncomment the -LP line to have page
%       numbers without the ``of ##'' last page reference)
%
% NOTE: uncomment the \pagestyle{empty} line to get rid of all page
%       numbers (make sure includefoot is commented out above)
%
\usepackage{fancyhdr,lastpage}
\pagestyle{fancy}
%\pagestyle{empty}      % Uncomment this to get rid of page numbers (no poner el número de las páginas)
\fancyhf{}\renewcommand{\headrulewidth}{0pt}
\fancyfootoffset{\marginparsep+\marginparwidth}
\newlength{\footpageshift}
\setlength{\footpageshift}
          {0.5\textwidth+0.5\marginparsep+0.5\marginparwidth-2in}
\lfoot{\hspace{\footpageshift}%
       \parbox{4in}{\, \hfill %
                    \arabic{page} of \protect\pageref*{LastPage} % +LP
%                    \arabic{page}                               % -LP
                    \hfill \,}}

% Finally, give us PDF bookmarks
\usepackage{color,hyperref}
\definecolor{darkblue}{rgb}{0.0,0.0,0.3}
\hypersetup{colorlinks,breaklinks,
            linkcolor=darkblue,urlcolor=darkblue,
            anchorcolor=darkblue,citecolor=darkblue}

%%%%%%%%%%%%%%%%%%%%%%%% End Document Setup %%%%%%%%%%%%%%%%%%%%%%%%%%%%


%%%%%%%%%%%%%%%%%%%%%%%%%%% Helper Commands %%%%%%%%%%%%%%%%%%%%%%%%%%%%

% The title (name) with a horizontal rule under it
% (optional argument typesets an object right-justified across from name
%  as well)
%
% Usage: \makeheading{name}
%        OR
%        \makeheading[right_object]{name}
%
% Place at top of document. It should be the first thing.
% If ``right_object'' is provided in the square-braced optional
% argument, it will be right justified on the same line as ``name'' at
% the top of the CV. For example:
%
%       \makeheading[\emph{Curriculum vitae}]{Your Name}
%
% will put an emphasized ``Curriculum vitae'' at the top of the document
% as a title. Likewise, a picture could be included:
%
%   \makeheading[\includegraphics[height=1.5in]{my_name}]{Manuel Doncel Martos}
%
% the picture will be flush right across from the name.
\newcommand{\makeheading}[2][]%
        {\hspace*{-\marginparsep minus \marginparwidth}%
         \begin{minipage}[t]{\textwidth+\marginparwidth+\marginparsep}%
             {\large \bfseries #2 \hfill #1}\\[-0.15\baselineskip]%
                 \rule{\columnwidth}{1pt}%
         \end{minipage}}



% The section headings
%
% Usage: \section{section name}
%
% Follow this section IMMEDIATELY with the first line of the section
% text. Do not put whitespace in between. That is, do this:
%
%       \section{My Information}
%       Here is my information.
%
% and NOT this:
%
%       \section{My Information}
%
%       Here is my information.
%
% Otherwise the top of the section header will not line up with the top
% of the section. Of course, using a single comment character (%) on
% empty lines allows for the function of the first example with the
% readability of the second example.
\renewcommand{\section}[2]%
        {\pagebreak[3]\vspace{1.3\baselineskip}%
         \phantomsection\addcontentsline{toc}{section}{#1}%
         \hspace{0in}%
         \marginpar{
         \raggedright \scshape #1}#2}

% An itemize-style list with lots of space between items
\newenvironment{outerlist}[1][\enskip\textbullet]%
        {\begin{itemize}[#1]}{\end{itemize}%
         \vspace{-.6\baselineskip}}

% An environment IDENTICAL to outerlist that has better pre-list spacing
% when used as the first thing in a \section
\newenvironment{lonelist}[1][\enskip\textbullet]%
        {\vspace{-\baselineskip}\begin{list}{#1}{%
        \setlength{\partopsep}{0pt}%
        \setlength{\topsep}{0pt}}}
        {\end{list}\vspace{-.6\baselineskip}}

% An itemize-style list with little space between items
\newenvironment{innerlist}[1][\enskip\textbullet]%
        {\begin{compactitem}[#1]}{\end{compactitem}}

% An environment IDENTICAL to innerlist that has better pre-list spacing
% when used as the first thing in a \section
\newenvironment{loneinnerlist}[1][\enskip\textbullet]%
        {\vspace{-\baselineskip}\begin{compactitem}[#1]}
        {\end{compactitem}\vspace{-.6\baselineskip}}

% To add some paragraph space between lines.
% This also tells LaTeX to preferably break a page on one of these gaps
% if there is a needed pagebreak nearby.
\newcommand{\blankline}{\quad\pagebreak[3]}
\newcommand{\halfblankline}{\quad\vspace{-0.5\baselineskip}\pagebreak[3]}

% Uses hyperref to link DOI
\newcommand\doilink[1]{\href{http://dx.doi.org/#1}{#1}}
\newcommand\doi[1]{doi:\doilink{#1}}

% For \url{SOME_URL}, links SOME_URL to the url SOME_URL
\providecommand*\url[1]{\href{#1}{#1}}
% Same as above, but pretty-prints SOME_URL in teletype fixed-width font
\renewcommand*\url[1]{\href{#1}{\texttt{#1}}}

% For \email{ADDRESS}, links ADDRESS to the url mailto:ADDRESS
\providecommand*\email[1]{\href{mailto:#1}{#1}}
% Same as above, but pretty-prints ADDRESS in teletype fixed-width font
%\renewcommand*\email[1]{\href{mailto:#1}{\texttt{#1}}}

%\providecommand\BibTeX{{\rm B\kern-.05em{\sc i\kern-.025em b}\kern-.08em
%    T\kern-.1667em\lower.7ex\hbox{E}\kern-.125emX}}
%\providecommand\BibTeX{{\rm B\kern-.05em{\sc i\kern-.025em b}\kern-.08em
%    \TeX}}
\providecommand\BibTeX{{B\kern-.05em{\sc i\kern-.025em b}\kern-.08em
    \TeX}}
\providecommand\Matlab{\textsc{Matlab}}

%%%%%%%%%%%%%%%%%%%%%%%% End Helper Commands %%%%%%%%%%%%%%%%%%%%%%%%%%%

%%%%%%%%%%%%%%%%%%%%%%%%% Begin CV Document %%%%%%%%%%%%%%%%%%%%%%%%%%%%

\begin{document}
% \makeheading{Manuel Doncel Martos, 34} % por si se quiere sin foto
\makeheading{\includegraphics[height=1.3in, width=1.1in]{profile.jpg}{Manuel Doncel Martos, 36}}

\section{Contact information}
%
% NOTE: Mind where the & separators and \\ breaks are in the following
%       table.
%
% ALSO: \rcollength is the width of the right column of the table
%       (adjust it to your liking; default is 1.85in).
%
\newlength{\rcollength}\setlength{\rcollength}{3.00 in}%
%
\begin{tabular}[t]{@{}p{\textwidth-\rcollength}p{\rcollength}}
George Gershwinlaan 381 & \textit{Mobile Phone Number:} +31 683 990 836 \\
1082 MT Amsterdam, North Holland, Netherlands & \textit{e-mail:} \email{manueldoncelmartos@gmail.com}\\
\end{tabular}

\section{Social Network}
%
\begin{outerlist}
\item[] \twitter ~ \href{http://www.twitter.com/manueldoncel}{@manueldoncel}
\item[] \textit{Skype}: ~ manuelarte
\item[] \linkedin \href{http://es.linkedin.com/in/manueldoncelmartos/en}{http://es.linkedin.com/in/manueldoncelmartos}
\item[] GitHub \href{https://github.com/manuelarte}{https://github.com/manuelarte}
\item[] Stackoverflow: \href{http://stackoverflow.com/users/2613670/manuelarte}{http://stackoverflow.com/users/2613670/manuelarte}
\end{outerlist}

\section{About me}
I am a Electrical Engineer with more than six years of experience as a Java Developer.
I consider myself an Object Oriented Programming and Scrum Agile Fan who tries to keep him up-to-date with the new technologies that appear. Continued learning, mainly using the Online Courses you can find in web pages like Coursera or edX, is a must for me and, also by developing my own projects.
I am looking to expand my experience and being happy to travel to achieve this.

\section{Career}
%
\includegraphics[scale=0.1]{companies/sytac.png} \href{https://www.sytac.op/}{\textbf{Sytac}},
Haarlem, Netherlands
\begin{outerlist}
\item[] \textit{Java Senior Developer} in Sytac%
        \hfill \textbf{July 2019 - Present}

I have worked in clients like Backbbase and KLM.
\begin{innerlist}
\item KLM: Working in a multi-project team on charge of develop the backend and frontend of webapps for internal use. Responsible for the full lifecycle of our projects, from development to deployment.

\item Backbase: Helping Backbase to integrate its product with a new customer in the pre-sales department. The result of the project was shown in Backbase Connect.

\end{innerlist}

\end{outerlist}

\includegraphics[scale=0.1]{companies/bux.jpg} \href{http://www.getbux.com/}{\textbf{BUX}},
Amsterdam, Netherlands
\begin{outerlist}
\item[] \textit{Java Senior Developer} in BUX%
        \hfill \textbf{April 2018 - June 2019}

Working as a Senior Software Developer in a microservice infrastructure in the service on charge of facilitating the conversion from a demo account to a real account. 
This service needs integration with several third parties with different APIs and different ways of responding. In this project I was also in charge of deployments to production and hotfixes.
\begin{innerlist}
\item Technologies: Spring Boot, Spring Data Jpa, RabbitMq, Hibernate, MySQL, Kubernetes, Google Cloud.
\end{innerlist}
\end{outerlist}

\includegraphics[scale=0.1]{companies/powerhouse.png} \href{http://www.powerhouse.nl/}{\textbf{Powerhouse}},
Amsterdam, Netherlands
\begin{outerlist}
\item[] \textit{Java Innovative Developer} in Inni%
        \hfill \textbf{November 2016 - April 2018}

Working as a Software Developer and Scrum Master in a multitenacy microservice event driven project to be used by electrical companies from different countries:
\begin{innerlist}
\item Technologies: Spring Boot, Spring Data Jpa, RabbitMq, Hibernate, MsSQL, Angular4.
\end{innerlist}
\item[] \textit{Java Innovative Developer} in Unity%
        \hfill \textbf{February 2016 - November 2016}

Working in a tool to be used by the Essent Zakelijke employees for contracting and billing:
\begin{innerlist}
\item Backend tasks (Spring, Spring MVC, Spring Security) and some small front end (Angular JS) tasks in an Agile Scrum environment.
\item Testing, focused on Unit Testing, using JUnit as the testing framework and Selenium for Functional Testing.
\end{innerlist}
\end{outerlist}

\includegraphics[scale=0.3]{companies/ericsson2.png} \href{http://www.ericsson.com/}{\textbf{Ericsson}},
Andalusia Technology Park, Malaga
\begin{outerlist}
\item[] \textit{Developer}%
        \hfill \textbf{June 2014 - February 2016}

Working in the \href{http://www.ericsson.com/ourportfolio/products/son-optimization-manager}{\textbf{SON-OM}} tool:
\begin{innerlist}
\item Experience as a Scrum Master.
\item Design and validation of simulators and real algorithms networks. Develpment of tools for optimization of the mobile multi-technology networks, emphasised in LTE networks.
\item Design and implementation of algorithms using pure Java as the programming language with a Scrum methodology.
\item Testing, focused on Unit Testing, using JUnit as the testing framework.
\item Project for AT\&T in the United States from Oct 2014 to Mar 2015. Focusing in performance (concurrency).
\end{innerlist}
\end{outerlist}

\includegraphics[scale=0.2]{companies/uma.jpg} \href{http://www.uma.es/}{\textbf{University of Malaga}},
Andalusia Technology Park, Malaga
\begin{outerlist}
\item[] \textit{Researcher}%
        \hfill \textbf{February 2013 - June 2014}

Working for Ericsson in:
\begin{innerlist}
\item Design and validation of simulators and real algorithms networks and tools of optimization of mobile multi-technology networks, emphasised in LTE networks.
\item Design and implementation of optimization tools using Java as the programming language using Scrum methodology.
\item Testing, focused on unit testing, using JUnit as the testing framework.
\end{innerlist}
\end{outerlist}

\includegraphics[scale=0.1]{companies/everis.jpg} \href{http://www.everis.es/}{\textbf{Everis}},
Madrid, Madrid
\begin{outerlist}

\item[] \textit{IT Solutions Assistant}%
        \hfill \textbf{October 2012 - February 2013}
\begin{innerlist}
\item Building a project from scratch for Vodafone to sign digital contracts usign \href{https://logalty.com/en/}{Logalty}.
\item Analyzed and developed Java Applications in a Scum environment.
\item Created web service with Apache Axis and XMLBeans.
\item Checked and Debugged Java Application, along with JUnit testing.
\end{innerlist}%
\end{outerlist}

\includegraphics[scale=0.1]{companies/laprimera.jpg} \href{http://www.laprimera.net/}{\textbf{LaPrimera.net}},
Ubeda, Jaen
\begin{outerlist}

\item[] \textit{IT Help Desk}%
        \hfill \textbf{January 2010 to June 2010}
\begin{innerlist}
\item Installated and launched Libre Software Virtual Stores, such as Prestashop; Content Management System, such as Joomla; and various Blogs and Forums.
\item Maintained and configured Virtual Hostings.
\item Helpdesk, hosting customization and modification.
\end{innerlist}

\end{outerlist}

%\halfblankline
%
%\href{http://www.uc3m.es/}{\textbf{Becario de la Universidad Carlos III de Madrid}},
%\begin{outerlist}
%
%\item[] \textit{}%
%        \hfill \textbf{Octubre 2005 a Julio 2006}
%\begin{innerlist}
%\item Árbitro de la liga interna de fútbol sala
%\item Becario de recepción para la residencia de estudiantes Fernando Abril Martorell
%\end{innerlist}
%
%\end{outerlist}

%\section{Education}
%
\href{http://www.ujaen.es/}{\textbf{University of Jaén}},
Linares, Jaen
\begin{outerlist}

\item[] \textit{Master of Science in Telecommunication Engineering}%
		\hfill \\ \textbf{October 2010 - September 2012}
		\begin{innerlist}
        %\item Only the Final Degree Project left
        \item Average grade 8.54 out of 10
        \item Final Degree Project: \emph{Implementation and Evaluation of a Robust Estimator of the Difference in Arrival Time for Reverberant Environments}.
        \item Tutor:
              \href{http://www4.ujaen.es/~pjreche}
                   {Pedro Jesus Reche Lopez}
%        \item Knowledge Area: Signal Theory
	    \end{innerlist}

\item[]  \textit{Bachelor of Science in Telecommunication Engineering: Telematics Speciality}%
		\hfill \\ \textbf{October 2007 to Febraury 2010}
        \begin{innerlist}
        \item Final Degree Project: \emph{Adaptative Environmental Noise Cancellation for Mobile Communications in Cars}
        \item Tutor:
              \href{http://www4.ujaen.es/~pvera}
                   {Pedro Vera Candeas}
%        \item Knowledge Area: Signal Theory.
        \end{innerlist}

\end{outerlist}%fin de información académica, por si no quieres poner lo de la Carlos III

%\halfblankline
%
%\href{http://www.uc3m.es/}{\textbf{Universidad Carlos III de Madrid}},
%Leganés, Madrid
%\begin{outerlist}
%
%\item[] \textit{Ingeniería de Telecomunicación}%
%		\hfill \textbf{Octubre 2004 a Septiembre 2007}
%		\begin{innerlist}
%        \item Primer Curso completado
%	    \end{innerlist}

%\end{outerlist}%fin de información académica

%\section{Courses}
%
e-learning
\begin{outerlist}
\item[] \href{http://www.edxonline.org/}{\edX}%
		%\hfill \textbf{March 2012 - June 2012}
		\begin{innerlist}
		\item MITx 8.02x: \textit{Electricity and Magnetism}. \href{https://s3.amazonaws.com/verify.edx.org/downloads/ff3a42be25c24842894e6e7b41bf87f5/Certificate.pdf}{Grade: \textbf{A}} \\ \textbf{February 2013 - July 2013}.
		\item BerkeleyX CS184.1x: \textit{Foundation of Computer Graphics}.  \href{https://s3.amazonaws.com/verify.edx.org/downloads/3c023624e80545c88d4dc0c3c8d1a785/Certificate.pdf}{Grade: \textbf{Pass}}\\ \textbf{November 2012 - December 2012}.		
		\item MITx 6.00x: \textit{Introduction to Computer Science and Programming}. \href{https://s3.amazonaws.com/verify.edx.org/downloads/17b57e8344e4469898ebf09779c58d14/Certificate.pdf}{Grade: \textbf{A}}. \\ \textbf{October 2012 - January 2013}.
		\item BerkeleyX CS169.1x: \textit{Software As A Service}. \href{https://s3.amazonaws.com/verify.edx.org/downloads/0883024d432241f6b3f85e5880c10d68/Certificate.pdf}{Grade: \textbf{Pass}}\\ \textbf{October 2012 - November 2012}.		
		\item MITx 6.002x: \textit{Circuits and Electronics}. \href{https://s3.amazonaws.com/verify.edxonline.org/downloads/058c25ccd2b14673955e777c6d51fe82/6002x_Certificate-58332.pdf}{Grade: \textbf{A}}. \\ \textbf{March 2012 - June 2012}
		\end{innerlist}
\end{outerlist}%fin de información académica
\begin{outerlist}
\item[] \href{https://www.udacity.com/}{\textbf{ \textcolor{udacity}{Udacity} }}%
		%\hfill \textbf{March 2012 - June 2012}
		\begin{innerlist}
		\item CS101: Building a Search Engine. Grade: \textbf{Highest Distinction}.
		\item Differential Equations in Action.
		\end{innerlist}
\end{outerlist}

\begin{outerlist}
\item[] \href{http://www.coursera.org/}{Coursera}%
		%\hfill \textbf{March 2012 - June 2012}
		\begin{innerlist}
		\item \textit{Functional Programming Principles in Scala}. \href{https://www.coursera.org/course/progfun}. \\ \textbf{May 2014 - August 2014}
		\item \textit{Think Again: How To Reason and Argue!} \href{https://www.coursera.org/maestro/api/certificate/get_certificate?course_id=226}{Grade: \textbf{87.2\%}}.\\ \textbf{November 2012 - February 2013}
		\item \textit{Learn To Program: Crafting Quality Code} \href{https://www.coursera.org/maestro/api/certificate/get_certificate?course_id=254}{Grade: \textbf{97.9\%}}. \\ \textbf{February 2013 - April 2013}		
		\item \textit{Computer Science 101}. \href{https://www.coursera.org/maestro/api/certificate/get_certificate?course_id=52}{Grade: \textbf{100\%}}. \\ \textbf{April 2012 - June 2012}
		\end{innerlist}
\end{outerlist}

%fin de información académica

% \section{Hardware and Software Skills}
%%
\textbf{Digital and Analogic Electronics:}
%%
\begin{innerlist}
   \item \textit{Microprocessors}: Microchip PIC, ATmega MCU's, Arduino, Raspberry PI.
   \item \textit{CAD}: Cadence OrCAD, NI Multisim, SPICE, Ngspice, LTspice, Microwave Office.
   \item \textit{DSP}: DSP56002 and C5515 of Texas Instrument.
\end{innerlist}
%
%\halfblankline
%
%Embedded and Real-time Systems:
%%
%\begin{innerlist}
%    \item Software and hardware development with several MCU and
%        DSP platforms (e.g., Motorola MCU's, Texas Instruments DSP's, Atmel
%        ATmega MCU's, Microchip PIC MCU's, and others)
%\end{innerlist}
%
%\halfblankline
%
%Instrumentation, Control, Data Acquisition, Test, and Measurement:
%%
%\begin{innerlist}
%    \item \href{http://www.dspaceinc.com/}{dSPACE} hardware (e.g.,
%        RTI1104) and Control Desk software,
%        \href{http://www.mathworks.com/products/simulink/}{Simulink},
%        \href{http://www.ni.com/}{LabVIEW} and other
%        \href{http://www.ni.com}{National Instruments}
%        control and data acquisition hardware and software (e.g., MIO,
%        SMIO, DSA, DMM, and others), Hewlett-Packard and Agilent
%        bench-top equipment
%\end{innerlist}
%
%\halfblankline
%
\textbf{Computing and Computer Programming:}
%%
\begin{innerlist}
	\item \textit{Office}: Microsoft Word, Excel, Project, \LaTeX{} (most of the office suites for both Windows and UNIX).
	\item \textit{Operating System}: Microsoft Windows Family, Linux, and other UNIX brunch.
    \item C, C$++$, Cuda C, Java, JUnit, JavaScript, Python, Ruby, Perl, PHP, UNIX shell scripting, GNU make, SQL, MySQL, \Matlab, Mathematica.
\end{innerlist}
%
%\halfblankline
%
%Version Control and Software Configuration Management:
%%
%\begin{innerlist}
%    \item DVCS (Mercurial/MQ, Git/StGit), VCS (RCS, CVS, SVN, SCCS), and
%        others
%\end{innerlist}
%
%\halfblankline
%
%\href{http://www.mathworks.com/products/matlab/}{\Matlab} skill set:
%%
%\begin{innerlist}
%    \item Linear algebra, Fourier transforms, Monte Carlo analysis,
%        nonlinear numerical methods, polynomials, statistics,
%        $N$-dimensional filters, visualization
%
%    \item Toolboxes: communications, control system, filter design,
%        genetic algorithm and direct search, signal processing, system
%        identification
%\end{innerlist}
%
%\halfblankline
%
%Software Verification:
%%
%\begin{innerlist}
%    \item KeY, PRISM, KeYmaera
%\end{innerlist}
%
%\halfblankline
%
\textbf{Local Area Network and Internet}

\begin{innerlist}
    \item Networking (UDP, TCP, ARP, DNS, SNMP), 802.XX, WiMAX, ZigBee, Services (Asterisk, Apache, SQL, POP, IMAP, SMTP, SOAP), Internet (HTTP, XML, etc)
    \item Cisco, Linksys, 3Com.
\end{innerlist}

\textbf{Mobile Communication}

\begin{innerlist}
	\item GSM, GPRS, WCDMA, UMTS, LTE.
	\item Dimensioning of 2G and 3G networks.
	\item Using programs like Momentos or Sirenet.
\end{innerlist}
%
%\halfblankline
%
%Productivity Applications:
%%
%\begin{innerlist}
%    \item \TeX{} (\LaTeX{}, \BibTeX{}, PSTricks), Vim,
%        most common productivity packages (for Windows, OS X, and Linux
%        platforms)
%\end{innerlist}
%
%\halfblankline
%
%Operating Systems:
%%
%\begin{innerlist}
%    \item Microsoft Windows family, Linux, y otras variantes UNIX.
%\end{innerlist}

%FIN DE LA SECCIÓN DE HABILIDADES

%\section{Abilities}
%
Languagues
\begin{outerlist}

\item[] \textit{Spanish (native language)}%

\item[]  \textit{English}%
        \begin{innerlist}
        \item Advanced understanding.
        \item Advanced reading 
		\item Fluent communication skills.
        \item First Certificate in English (B2). 
        \end{innerlist}
        
\item[]  \textit{French}%
        \begin{innerlist}
        \item Basic skills in understanding, reading and communication skills.
        \end{innerlist}

\end{outerlist}%fin de información académica

\halfblankline

%Capacidades y comunicación
Capabilities and Strengths
\begin{outerlist}

\item[] \textit{Capabilities and social skills}%
		\begin{innerlist}
        \item Teamwork.
        \item Concentration.
        \item Self development.
        \item Responsibility.
        \item Communication.
	    \end{innerlist}

%\item[]  \textit{Competitividad y capacidades organizativas}%
        \begin{innerlist}
        \item B Driver license.
        \item Availability (geographic and time).
        \end{innerlist}
\end{outerlist}%fin de información académica

%\section{Hobbies}
%Deportes
%\begin{outerlist}
%
%\item[] \textit{Fútbol y Fútbol Sala}%
%
%\begin{innerlist}
%\item Miembro del equipo de Fútbol Sala de la Universidad (2005-2007)
%\item Jugador de Preferente de Fútbol Sala (2006-2007)
%\end{innerlist}
%
%\item[] \textit{Baloncesto, Padel}%
%%\item[] \textit{Padel}%
%
%\end{outerlist}
%
%\halfblankline
%
%Telecomunicaciones
%\begin{outerlist}
%
%\item[] \textit{Electrónica}%
%\begin{innerlist}
%    \item Desarrollo de proyectos Hardware libre con el proyecto Arduino
%    \item Desarrollo de proyectos con la gama de microcontroladores PIC
%\end{innerlist}
%
%\item[] \textit{Voz sobre IP}%
%\begin{innerlist}
%    \item VoIP con el proyecto Asterisk
%\end{innerlist}
%\end{outerlist}

\end{document}

%%%%%%%%%%%%%%%%%%%%%%%%%% End CV Document %%%%%%%%%%%%%%%%%%%%%%%%%%%%%
